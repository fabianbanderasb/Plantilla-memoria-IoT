%%%%%%%%%%%%%%%%%%%%%%%%%%%%%%%%%%%%%%%%%%%%%%%%%%%%%%%%%%%%%%%%%%%%%%%%%%%%%%%%
%
% Template license:
% CC BY-NC-SA 3.0 (http://creativecommons.org/licenses/by-nc-sa/3.0/)
%
%%%%%%%%%%%%%%%%%%%%%%%%%%%%%%%%%%%%%%%%%%%%%%%%%%%%%%%%%%%%%%%%%%%%%%%%%%%%%%%%

%----------------------------------------------------------------------------------------
%	PACKAGES AND OTHER DOCUMENT CONFIGURATIONS
%----------------------------------------------------------------------------------------

\documentclass[
11pt, % The default document font size, options: 10pt, 11pt, 12pt
%oneside, % Two side (alternating margins) for binding by default, uncomment to switch to one side
%chapterinoneline,% Have the chapter title next to the number in one single line
spanish,
singlespacing, % Single line spacing, alternatives: onehalfspacing or doublespacing
%draft, % Uncomment to enable draft mode (no pictures, no links, overfull hboxes indicated)
%nolistspacing, % If the document is onehalfspacing or doublespacing, uncomment this to set spacing in lists to single
%liststotoc, % Uncomment to add the list of figures/tables/etc to the table of contents
%toctotoc, % Uncomment to add the main table of contents to the table of contents
parskip, % Uncomment to add space between paragraphs
%codirector, % Uncomment to add a codirector to the title page
headsepline, % Uncomment to get a line under the header
]{MastersDoctoralThesis} % The class file specifying the document structure



%----------------------------------------------------------------------------------------
%	INFORMACIÓN DE LA MEMORIA
%----------------------------------------------------------------------------------------

\thesistitle{Desarrollo de aplicación para empleados SER\&PRO Services \& Products S.A. con notificación push} % El títulos de la memoria, se usa en la carátula y se puede usar el cualquier lugar del documento con el comando \ttitle

% Nombre del posgrado, se usa en la carátula y se puede usar el cualquier lugar del documento con el comando \degreename
\posgrado{Carrera de Especialización en Internet de las Cosas} 
%\posgrado{Carrera de Especialización en Internet de las Cosas} 
%\posgrado{Carrera de Especialización en Intelegencia Artificial}
%\posgrado{Maestría en Sistemas Embebidos} 
%\posgrado{Maestría en Internet de las cosas}

\author{Ing. 	Fabián Alejandro Banderas Benítez} % Tu nombre, se usa en la carátula y se puede usar el cualquier lugar del documento con el comando \authorname

\director{Mg. Ing. Yoel Yamil López} % El nombre del director, se usa en la carátula y se puede usar el cualquier lugar del documento con el comando \dirname
%\codirector{Nombre del codirector (pertenencia)} % El nombre del codirector si lo hubiera, se usa en la carátula y se puede usar el cualquier lugar del documento con el comando \codirname.  Para activar este campo se debe descomentar la opción "codirector" en el comando \documentclass, línea 23.

\juradoUNO{Nombre del jurado 1 (pertenencia)} % Nombre y pertenencia del un jurado se usa en la carátula y se puede usar el cualquier lugar del documento con el comando \jur1name
\juradoDOS{Nombre del jurado 2 (pertenencia)} % Nombre y pertenencia del un jurado se usa en la carátula y se puede usar el cualquier lugar del documento con el comando \jur2name
\juradoTRES{Nombre del jurado 3 (pertenencia)} % Nombre y pertenencia del un jurado se usa en la carátula y se puede usar el cualquier lugar del documento con el comando \jur3name

%\ciudad{Ciudad Autónoma de Buenos Aires}
\ciudad{Ciudad Autónoma de Buenos Aires}

\fechaINICIO{agosto de 2022}
\fechaFINAL{junio de 2023}


\keywords{Internet de las Cosas, FIUBA} % Keywords for your thesis, print it elsewhere with \keywordnames


\begin{document}


\frontmatter % Use roman page numbering style (i, ii, iii, iv...) for the pre-content pages

\pagestyle{plain} % Default to the plain heading style until the thesis style is called for the body content


%----------------------------------------------------------------------------------------
%	RESUMEN - ABSTRACT 
%----------------------------------------------------------------------------------------

\begin{abstract}
\addchaptertocentry{\abstractname} % Add the abstract to the table of contents
%
%The Thesis Abstract is written here (and usually kept to just this page). The page is kept centered vertically so can expand into the blank space above the title too\ldots
\centering

La presente memoria de trabajo describre el desarrollo de una aplicación para empleados SER\&PRO Services \& Products S.A con notificación push. Se enfocó en desarrollar una aplicación que permita a los usuarios recibir mensajes informativos, personalizados con llamadas a la acción. La metodología SCRUM fue utilizada durante el proceso de desarrollo, por el Ingeniero Fabián Banderas liderando el equipo y el Máster Ingeniero Yoel Lopez como director del proyecto. El valor del proyecto radica en la mejora de la comunicación entre la empresa y sus empleados a través de una herramienta tecnológica, innovadora y eficiente. 

En la memoria del proyecto, el lector encontrará detalles sobre el proceso de desarrollo y la implementación de la aplicación. Se aplicaron conocimientos en desarrollo de aplicaciones móviles y metodologías ágiles para lograr los objetivos del proyecto.

\end{abstract}

%----------------------------------------------------------------------------------------
%	CONTENIDO DE LA MEMORIA  - AGRADECIMIENTOS
%----------------------------------------------------------------------------------------

\begin{acknowledgements}
%\addchaptertocentry{\acknowledgementname} % Descomentando esta línea se puede agregar los agradecimientos al índice
\vspace{1.5cm}

Es un honor para mí compartir este camino que he recorrido, que ha sido guiado por las mejores influencias de mi vida: Dios y mis padres.

Durante mi paso por la Facultad de Ingeniería de la Universidad de Buenos Aires, he tenido la dicha de conocer a personas muy valiosas, de quienes agradezco infinitamente la sabiduría, habilidades, conocimientos y valores que se han fortalecido en mí a lo largo de todo este tiempo.
 

\end{acknowledgements}

%----------------------------------------------------------------------------------------
%	LISTA DE CONTENIDOS/FIGURAS/TABLAS
%----------------------------------------------------------------------------------------

\tableofcontents % Prints the main table of contents

\listoffigures % Prints the list of figures

\listoftables % Prints the list of tables


%----------------------------------------------------------------------------------------
%	CONTENIDO DE LA MEMORIA  - DEDICATORIA
%----------------------------------------------------------------------------------------


\dedicatory{\textbf{Este trabajo está dedicado a mi Dios, quien guía mi camino diariamente y me da la fuerza para enfrentar las circunstancias de la vida con valentía sin perder mi esencia ni valores.\\
Agradezco a mi familia, compañeros y docentes por acompañarme en este sendero y ser una fuente de luz y apoyo constante.\\
Gracias por confiar en mí y dedicar tiempo a mi crecimiento personal.}}


%----------------------------------------------------------------------------------------
%	CONTENIDO DE LA MEMORIA  - CAPÍTULOS
%----------------------------------------------------------------------------------------

\mainmatter % Begin numeric (1,2,3...) page numbering

\pagestyle{thesis} % Return the page headers back to the "thesis" style

% Incluir los capítulos como archivos separados desde la carpeta Chapters

% Chapter 1

\chapter{Introducción general} % Main chapter title

\label{Chapter1} % For referencing the chapter elsewhere, use \ref{Chapter1} 
\label{IntroGeneral}

%----------------------------------------------------------------------------------------

% Define some commands to keep the formatting separated from the content 
\newcommand{\keyword}[1]{\textbf{#1}}
\newcommand{\tabhead}[1]{\textbf{#1}}
\newcommand{\code}[1]{\texttt{#1}}
\newcommand{\file}[1]{\texttt{\bfseries#1}}
\newcommand{\option}[1]{\texttt{\itshape#1}}
\newcommand{\grados}{$^{\circ}$}

%----------------------------------------------------------------------------------------

%\section{Introducción}

%----------------------------------------------------------------------------------------
\section{Motivación}

%----------------------------------------------------------------------------------------

\section{Planteo conceptual}

%----------------------------------------------------------------------------------------

\section{Objetivos y alcance}

\subsection{Principal}

\subsection{Específicos}

%----------------------------------------------------------------------------------------

\section{Estado del arte}

%----------------------------------------------------------------------------------------

\chapter{Introducción específica} % Main chapter title

\label{Chapter2}

%----------------------------------------------------------------------------------------
%	SECTION 1
%----------------------------------------------------------------------------------------
Todos los capítulos deben comenzar con un breve párrafo introductorio que indique cuál es el contenido que se encontrará al leerlo.  La redacción sobre el contenido de la memoria debe hacerse en presente y todo lo referido al proyecto en pasado, siempre de modo impersonal.

\section{Principales tecnologías}
\label{sec:ejemplo}


%----------------------------------------------------------------------------------------
%	SECTION 2
%----------------------------------------------------------------------------------------
Todos los capítulos deben comenzar con un breve párrafo introductorio que indique cuál es el contenido que se encontrará al leerlo.  La redacción sobre el contenido de la memoria debe hacerse en presente y todo lo referido al proyecto en pasado, siempre de modo impersonal.

\section{Módulos y componentes}

%----------------------------------------------------------------------------------------




 
\chapter{Diseño e implementación} % Main chapter title

\label{Chapter3} % Change X to a consecutive number; for referencing this chapter elsewhere, use \ref{ChapterX}

\definecolor{mygreen}{rgb}{0,0.6,0}
\definecolor{mygray}{rgb}{0.5,0.5,0.5}
\definecolor{mymauve}{rgb}{0.58,0,0.82}

%%%%%%%%%%%%%%%%%%%%%%%%%%%%%%%%%%%%%%%%%%%%%%%%%%%%%%%%%%%%%%%%%%%%%%%%%%%%%
% parámetros para configurar el formato del código en los entornos lstlisting
%%%%%%%%%%%%%%%%%%%%%%%%%%%%%%%%%%%%%%%%%%%%%%%%%%%%%%%%%%%%%%%%%%%%%%%%%%%%%
\lstset{ %
  backgroundcolor=\color{white},   % choose the background color; you must add \usepackage{color} or \usepackage{xcolor}
  basicstyle=\footnotesize,        % the size of the fonts that are used for the code
  breakatwhitespace=false,         % sets if automatic breaks should only happen at whitespace
  breaklines=true,                 % sets automatic line breaking
  captionpos=b,                    % sets the caption-position to bottom
  commentstyle=\color{mygreen},    % comment style
  deletekeywords={...},            % if you want to delete keywords from the given language
  %escapeinside={\%*}{*)},          % if you want to add LaTeX within your code
  %extendedchars=true,              % lets you use non-ASCII characters; for 8-bits encodings only, does not work with UTF-8
  %frame=single,	                % adds a frame around the code
  keepspaces=true,                 % keeps spaces in text, useful for keeping indentation of code (possibly needs columns=flexible)
  keywordstyle=\color{blue},       % keyword style
  language=[ANSI]C,                % the language of the code
  %otherkeywords={*,...},           % if you want to add more keywords to the set
  numbers=left,                    % where to put the line-numbers; possible values are (none, left, right)
  numbersep=5pt,                   % how far the line-numbers are from the code
  numberstyle=\tiny\color{mygray}, % the style that is used for the line-numbers
  rulecolor=\color{black},         % if not set, the frame-color may be changed on line-breaks within not-black text (e.g. comments (green here))
  showspaces=false,                % show spaces everywhere adding particular underscores; it overrides 'showstringspaces'
  showstringspaces=false,          % underline spaces within strings only
  showtabs=false,                  % show tabs within strings adding particular underscores
  stepnumber=1,                    % the step between two line-numbers. If it's 1, each line will be numbered
  stringstyle=\color{mymauve},     % string literal style
  tabsize=2,	                   % sets default tabsize to 2 spaces
  title=\lstname,                  % show the filename of files included with \lstinputlisting; also try caption instead of title
  morecomment=[s]{/*}{*/}
}


%----------------------------------------------------------------------------------------
%	SECTION 1
%----------------------------------------------------------------------------------------
\section{Arquitectura del sistema}
 

%----------------------------------------------------------------------------------------
%	SECTION 2
%----------------------------------------------------------------------------------------
Todos los capítulos deben comenzar con un breve párrafo introductorio que indique cuál es el contenido que se encontrará al leerlo.  La redacción sobre el contenido de la memoria debe hacerse en presente y todo lo referido al proyecto en pasado, siempre de modo impersonal.

\section{Detalle de módulos de hardware}

%----------------------------------------------------------------------------------------


%----------------------------------------------------------------------------------------
%	SECTION 3
%----------------------------------------------------------------------------------------
Todos los capítulos deben comenzar con un breve párrafo introductorio que indique cuál es el contenido que se encontrará al leerlo.  La redacción sobre el contenido de la memoria debe hacerse en presente y todo lo referido al proyecto en pasado, siempre de modo impersonal.

\section{Detalle de módulos de software}

%----------------------------------------------------------------------------------------


%----------------------------------------------------------------------------------------
%	SECTION 4
%----------------------------------------------------------------------------------------
Todos los capítulos deben comenzar con un breve párrafo introductorio que indique cuál es el contenido que se encontrará al leerlo.  La redacción sobre el contenido de la memoria debe hacerse en presente y todo lo referido al proyecto en pasado, siempre de modo impersonal.

\section{Interfaz del sistema}

%----------------------------------------------------------------------------------------



% Chapter Template

\chapter{Ensayos y resultados} % Main chapter title

\label{Chapter4} % Change X to a consecutive number; for referencing this chapter elsewhere, use \ref{ChapterX}

%----------------------------------------------------------------------------------------
%	SECTION 1
%----------------------------------------------------------------------------------------

\section{Detalle de pruebas realizadas (descripción del banco de pruebas)}
\label{sec:pruebasHW}

La idea de esta sección es explicar cómo se hicieron los ensayos, qué resultados se obtuvieron y analizarlos.


%----------------------------------------------------------------------------------------
%	SECTION 2
%----------------------------------------------------------------------------------------
Todos los capítulos deben comenzar con un breve párrafo introductorio que indique cuál es el contenido que se encontrará al leerlo.  La redacción sobre el contenido de la memoria debe hacerse en presente y todo lo referido al proyecto en pasado, siempre de modo impersonal.

\section{Pruebas unitarias}

%----------------------------------------------------------------------------------------


%----------------------------------------------------------------------------------------
%	SECTION 3
%----------------------------------------------------------------------------------------
Todos los capítulos deben comenzar con un breve párrafo introductorio que indique cuál es el contenido que se encontrará al leerlo.  La redacción sobre el contenido de la memoria debe hacerse en presente y todo lo referido al proyecto en pasado, siempre de modo impersonal.

\section{Pruebas del sistema}

%----------------------------------------------------------------------------------------


%----------------------------------------------------------------------------------------
%	SECTION 4
%----------------------------------------------------------------------------------------
Todos los capítulos deben comenzar con un breve párrafo introductorio que indique cuál es el contenido que se encontrará al leerlo.  La redacción sobre el contenido de la memoria debe hacerse en presente y todo lo referido al proyecto en pasado, siempre de modo impersonal.

\section{Pruebas de aceptación}

%----------------------------------------------------------------------------------------

%----------------------------------------------------------------------------------------
%	SECTION 5
%----------------------------------------------------------------------------------------
Todos los capítulos deben comenzar con un breve párrafo introductorio que indique cuál es el contenido que se encontrará al leerlo.  La redacción sobre el contenido de la memoria debe hacerse en presente y todo lo referido al proyecto en pasado, siempre de modo impersonal.

\section{Comparativa con otras soluciones del mercado}

%----------------------------------------------------------------------------------------

 
% Chapter Template

\chapter{Conclusiones} % Main chapter title

\label{Chapter5} % Change X to a consecutive number; for referencing this chapter elsewhere, use \ref{ChapterX}


%----------------------------------------------------------------------------------------

%----------------------------------------------------------------------------------------
%	SECTION 1
%----------------------------------------------------------------------------------------

\section{Resultados obtenidos}

La idea de esta sección es resaltar cuáles son los principales aportes del trabajo realizado y cómo se podría continuar. Debe ser especialmente breve y concisa. Es buena idea usar un listado para enumerar los logros obtenidos.


%----------------------------------------------------------------------------------------
%	SECTION 2
%----------------------------------------------------------------------------------------
\section{Trabajo futuro}

Acá se indica cómo se podría continuar el trabajo más adelante.
 

%----------------------------------------------------------------------------------------
%	CONTENIDO DE LA MEMORIA  - APÉNDICES
%----------------------------------------------------------------------------------------

\appendix % indicativo para indicarle a LaTeX los siguientes "capítulos" son apéndices

% Incluir los apéndices de la memoria como archivos separadas desde la carpeta Appendices
% Descomentar las líneas a medida que se escriben los apéndices

%% Appendix A

\chapter{Appendix Title Here} % Main appendix title

\label{AppendixA} % For referencing this appendix elsewhere, use \ref{AppendixA}

Write your Appendix content here.


%\include{Appendices/AppendixB}
%\include{Appendices/AppendixC}

%----------------------------------------------------------------------------------------
%	BIBLIOGRAPHY
%----------------------------------------------------------------------------------------

\Urlmuskip=0mu plus 1mu\relax
\raggedright
\printbibliography[heading=bibintoc]

%----------------------------------------------------------------------------------------

\end{document}  
